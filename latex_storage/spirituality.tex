\title{Программа рода (духовность)}

\description
Сфера духовности отвечает за наши стремления к духовности и проработки, 
связь с Богом и весь наш род и родовые программы.
\enddescription

\section{id="1"}{Январь}
Рожденные в январе – первопроходцы. Среди них есть много новых душ. 
В январе рождаются очень мудрые, сильные, целеустремлённые и выносливые 
люди. Перед ними будет много завершающихся задач с высокой сложностью 
и вибрациями.
Они приходят со следующей задачей перед родом: очистка кармы родных, 
кровных братьев или сестёр. У них могут быть конфликты с кровными 
братьями и сёстрами, но в этом случае кармически важен компромисс.
Важно заботиться, оберегать их и любить. Если нет родных, но есть 
двоюродные или сводные братья и сёстры, очень близкие с детства 
(даже если что-то пошло не так, и вы разругались), всё равно, 
Ваша задача - поддерживать с ними связь, общаться, помогать им. 
Вы для этих людей как путеводная звезда, они без Вас не справятся.
Самая большая проблема в Вашей прошлой жизни – эгоизм. Вы жили для 
себя и не делились. Поэтому в этом воплощении, отрабатывая программы 
рода, Вы должны отдавать, оберегать и заботиться о другом человеке. 
Так же возможно, у Вас нет связи с этим человеком, Вы не можете его 
найти. Тогда в Вашем поле должны быть люди слабее Вас, зависимые от 
Вас, которые по внутреннему ощущению для Вас как братья или сёстры 
и Вы должны помочь им реализоваться.
\endsection

\section{id="2"}{Февраль}
В прошлой жизни у человека, родившегося в феврале, была проблема 
с его родом. Он либо причинил боль кому-то из своего рода, отказался 
или покинул род. В этой жизни он должен оказаться в такой жизненной 
ситуации, где он будет очень сильно нуждаться от своего рода. 
Как правило, люди, рожденные в феврале, воспитываются бабушками 
или дедушками и редко проводят время с мамой. В большинстве случаев 
они недолюблены мамой. Вы можете постоянно оказываться в таких 
условиях, где Вам нужно проявлять заботу, любовь и внимание.
Программа февральских может прорабатываться в детстве с помощью 
заботы о животных.
Обязательно нужно прощать обиды, в том числе на свой род.
Учиться делиться, заботиться о женщинах своего рода (о маме, бабушке), 
помогать старикам.
\endsection

\section{id="3"}{Март}
Если человек родился в марте, это говорит о нарушениях традиции 
воспитания детей в роду.
Это про то, что женщины в роду не были реализованы. 
Первая проблема в роду – нежелание иметь детей. 
Вторая проблема – ребёнок растёт сам по себе, к нему проявляют 
мало внимания и заботы. 
Третья проблема – это мама-«наседка», проявляющая чрезмерный 
контроль и переживающая за ребёнок. 
Ваша задача – выбрать противоположную сторону в воспитании детей 
и простить обиды на маму, не осуждать её. Нужно отделиться от мамы, 
не жить с ней, не зависеть от неё, выйти из-под её влияния. Но очень 
важно любить свою маму.
Ваша задача в этой жизни – реализоваться.
\endsection

\section{id="4"}{Апрель}
Люди, рожденные в апреле, имеют противоречивые отношения с отцом. 
Обязательно нужно наладить с ним контакт, если еще есть такая 
возможность. Успех человека в жизни формируется на любви, уважении 
и принятии своих родителей. Если этого нет, то не будет ни денег, 
ни отношений. Причём деньги – это отец, а отношения – это мама. 
Если у Вас маленький заработок – прорабатывайте отца и обиды на него. 
Если Вы не будете в хороших отношениях, то это скажется на Вашей карьере. 
Ваша задача – стать богаче и успешнее отца, превзойти его.
Ребенок, рождаясь в апреле, как бы становится на место своего отца, 
становится кормильцем в семье. Особенно это касается мужчин: Вам важно 
взять ответственность за свою семью на себя, Вам нельзя разрушать семью, 
Вы должны зарабатывать больше, чем Ваша жена. 
Женщина, рожденная в апреле, - Вы можете помогать семье, где Вы родились 
и выросли деньгами. Вы можете зарабатывать, но Вам важно переключать 
четвёрку в тройку, т.е. стать матерью, стать женой, стать женщиной, 
которая реализована. Тогда у Вас будет баланс. Иначе, если Вы заработаетесь, и
\endsection

\section{id="5"}{Май}
Ваша основная родовая программа – это проработать программу своей мамы. 
Вы для неё учитель. Она не была свободна в своей жизни. Она принимала 
свои решения нехотя. Она жила в ограничениях. Вы должны выбраться из 
этих ограничений. Вы должны жить свободную жизнь. Чтобы Ваша мама через 
Вас увидела, что может быть по-другому. И тогда Вы сломаете родовые 
сценарии. Вам важно наладить отношения через любовь, заботу и взаимопонимание.
\endsection

\section{id="8"}{Август}
Рожденный в августе ребёнок
\endsection

\section{id="9"}{Сентябрь}
В сентябре рождаются невероятно талантливые люди, которые имеют 
родовую задачу: раскрыть таланты свои и своего рода: таланты в 
творчестве, таланты в эзотерике и т. д. 
Ваша задача проявить их. Дать пользу миру через таланты рода. 
Это значит, что родители занимались не тем, чем должны были. 
Они не раскрыли свой талант и Вам нужно это сделать за них.
Обычно они проявляются до 5 лет, но можно их раскрыть после, 
вернувшись в детство и попробовав реализовать себя.
Невыполнение этого предназначения может привести человека в изоляцию, 
на него не будут обращать внимание и общаться. Он может терять близких 
людей, к которым привязан. Ему нужно не бояться выходить в массы, 
не стесняться славы, сцены, выступлений. Работать не там, где выгодно, 
а там, где хочется.
\endsection

\section{id="10"}{Октябрь}
Задача октябрьских людей – восстановить финансовую составляющую 
своего рода и увеличить его капитал. Эти люди должны зарабатывать 
большие деньги.
То, что было заработано в роду, нужно увеличить в разы и при этом 
не потратить, а сделать что-то для семьи, своих детей, оставить 
наследство, обеспечить родителям старость. Вы можете получить 
наследство, чтобы приумножить его и передавать дальше.
На людей, рожденных в октябре, ложится ответственность не только 
в финансовом плане, но и в воспитании правильного отношения между 
родственниками в вопросах имущества.
Вы должны построить свою династию!
\endsection

\section{id="11"}{Ноябрь}
Ноябрьские дети – это те, кто должен приумножить статус своего рода, 
прославить его.
Следует выбирать профессии, которые имеют высокий доход и уровень 
в обществе. Эти люди должны стать знаменитыми.
\endsection

\section{id="12"}{Декабрь}
У декабрьского ребёнка нарушены межличностные связи в роду, нет 
желания идти навстречу друг другу.
Как правило, мама не могла положиться на своего мужа, или что-то 
пошло не так между родителями и детьми.
Их задача – исправить такие взаимоотношения, скрепить отношения 
между родственниками. Они являются хорошими психологами на бытовом 
уровне, хорошо чувствуют людей. Такие люди часто отстраняются от 
семьи, не нуждаются в их помощи и советах. Для того, чтобы выполнить 
свою задачу, им необходимо не впадать в депрессию, не обижаться, 
не винить других. Важно следить за тем, чтобы не жертвовать собой 
и учиться просить помощи, так как у детей, рожденных в декабре, 
очень часто идет перекос: я буду вас спасать, я буду вам помогать, 
я буду вас обеспечивать, я буду вам все давать, я буду жить с вами, 
чтобы вам было лучше. И родители начинают болеть и страдать, и 
становиться всё хуже и хуже. Декабрьский ребёнок должен отсоединиться 
от родителей и помогать им только в крайнем случае. Чем больше энергии 
Вы им отдаете, тем слабее они становятся.
У 12-го аркана есть склонность к депрессии. Им важно научиться 
управлять своими эмоциями.
Депрессия у них на почве вины, а вина из-за того, что они что-то 
не додали родителям: не такие хорошие дети, не оправдали ожидания. 
Важно понять, что нужно быть хорошим в первую очередь для себя и 
тогда у Вас всё будет хорошо.
\endsection

\section{id="12"}{Декабрь}
У декабрьского ребёнка нарушены межличностные связи в роду, нет 
желания идти навстречу друг другу.
Как правило, мама не могла положиться на своего мужа, или что-то 
пошло не так между родителями и детьми.
Их задача – исправить такие взаимоотношения, скрепить отношения 
между родственниками. Они являются хорошими психологами на бытовом 
уровне, хорошо чувствуют людей. Такие люди часто отстраняются от 
семьи, не нуждаются в их помощи и советах. Для того, чтобы выполнить 
свою задачу, им необходимо не впадать в депрессию, не обижаться, 
не винить других. Важно следить за тем, чтобы не жертвовать собой 
и учиться просить помощи, так как у детей, рожденных в декабре, 
очень часто идет перекос: я буду вас спасать, я буду вам помогать, 
я буду вас обеспечивать, я буду вам все давать, я буду жить с вами, 
чтобы вам было лучше. И родители начинают болеть и страдать, и 
становиться всё хуже и хуже. Декабрьский ребёнок должен отсоединиться 
от родителей и помогать им только в крайнем случае. Чем больше энергии 
Вы им отдаете, тем слабее они становятся.
У 12-го аркана есть склонность к депрессии. Им важно научиться 
управлять своими эмоциями.
Депрессия у них на почве вины, а вина из-за того, что они что-то 
не додали родителям: не такие хорошие дети, не оправдали ожидания. 
Важно понять, что нужно быть хорошим в первую очередь для себя и 
тогда у Вас всё будет хорошо.
\endsection

\section{id="1_genus"}{Родовые программы}

\subsection{Предназначение:}
Быть лидером и делиться своими знаниями, управлять своим разумом, 
своей силой мысли и использовать это во благо, реализовать свой 
потенциал оратора и духовного проводника, опираясь на позитивное 
мышление и могущество своего слова.
\endsubsection

\subsection{Минусы:}
Неверие в себя, нежелание брать ответственность и лидерство в 
свои руки, апатия и лень; другое проявление – излишняя самонадеянность 
и завышенная самооценка, манипуляции другими людьми, преследование 
своих корыстных целей.
\endsubsection

\subsection{Рекомендации:}
\item Прокачивать свои лидерские качества и уверенность в себе – 
тренинги, выступления, курсы, психолог.
\item Развивать ораторский талант.
\item Вырабатывать позитивное мышление – основу материализации мыслей.
\endsubsection

\endsection

\section{id="2_genus"}{Родовые программы}

\subsection{Предназначение:}
Заботиться о других, принимая и даря безусловную любовь, для этого 
необходимо развивать свою интуицию и доверять ей, быть искренним 
и решительным, стать человеком, который сеет добро и исцеляет души. 
Стать таким образом проводником тайных знаний, служить людям, 
раскрыть целительские способности.
\endsubsection

\subsection{Минусы:}
Неумение прислушиваться и доверять себе, обидчивость и озлобленность 
на весь мир, скрытность и корысть, нежелание помогать другим.
Вас по жизни могут проверять на сплетничество, на женатых мужчин, 
любовные треугольники и т.д. Ваш дух будут искушать.
\endsubsection

\subsection{Рекомендации:}
\item Развивать интуицию и сверхчувствительность через различные 
практики, обучения.
\item Раскрывать свои целительские способности.
\item Помогать людям, проявлять искреннюю заботу, совершать добрые 
поступки просто так, без причин.
\item Проработать обиды через осознанность, психолога и т.п.
\item Рекомендуются медитации для успокоения разума, увеличения 
энергетической проводимости, отпускания обид, общего роста 
духовной составляющей и т.д.
\endsubsection

\endsection

\section{id="3_genus"}{Родовые программы}

\subsection{Предназначение:}
Для Вас программа – стать хорошей матерью (для женщин) или хорошим 
отцом (для мужчин). Развивать в себе женские/мужские энергии, стать 
родителем в широком смысле слова – человеком заботы и изобилия, 
проводником для раскрытия талантов. При этом демонстрируя свою 
уверенность, лёгкость и свободное течение энергии. Очень важно 
быть в хороших отношениях со своей мамой и сепарироваться от неё, 
не зависеть от неё.
\endsubsection

\subsection{Минусы:}
Высокомерие, конкуренция с мужчинами (для женщин), излишняя 
женственность и безответственность (для мужчин), нежелание 
отдавать и становиться родителем.
\endsubsection

\subsection{Рекомендации:}
\item Наладить и гармонизировать отношения с женщинами в роду и 
не только – без перекосов (ненависть/зависимость)
\item Учиться отдавать без самопожертвования из изобилия и 
внутренней гармонии.
\item Для женщин: развивать свою женственность, лёгкость – танцы, 
практики расслабления, медитации; не брать излишнюю ответственность 
на себя – прокачка доверия к мужчинам; вдохновлять своего избранника.
\item Для мужчин: брать на себя ответственность: за свою семью, 
коллектив и т.п.; занимать активную жизненную позицию; развивать 
мужественность и смелость, в т.ч. через спорт; не быть «маменьким 
сынком».
\endsubsection

\endsection

\section{id="4_genus"}{Родовые программы}

\subsection{Предназначение:}
Стать хорошим родителем, создать полноценную семью. Ценить эту 
семью, не разрушать ее, быть ответственным перед своими детьми.
4-й аркан указывает на рождение в семье сына.
\endsubsection

\subsection{Минусы:}
Подавление других людей, жёсткость, даже жестокость и авторитаризм. 
И обратное проявление – слабость, нежелание брать ответственность, 
апатия.
\endsubsection

\subsection{Рекомендации:}
\item Наладить отношения с мужским полом в семье, простить отца, 
если есть обиды.
\item Контролировать свои негативные качества и эмоции – властность, 
жестокость, агрессию.
\item Создавать полноценную семью с детьми.
\item Не впадать в апатию и лень, нужна активная жизненная позиция.
\item Для женщин: развивать женственность, смягчая проявление своих 
мужских качеств; не тянуть своего избранника, а довериться ему и 
предоставить возможность «рулить».
\item Для мужчин: стать настоящим мужчиной: и добытчиком, и решающим 
бытовые вопросы; заниматься спортом – это отличная прокачка необходимых 
вам мужских качеств.
\endsubsection

\endsection

\section{id="5_genus"}{Родовые программы}

\subsection{Предназначение:}
Стать учителем/наставником в самом широком смысле: передавать свои 
знания, быть примером, постоянно развиваться, почитать традиции, 
а также вносить в них новое видение. Используя свои ораторские 
навыки, распространять новые идеи в массы.
Чтить и уважать свой род.
\endsubsection

\subsection{Минусы:}
Осуждение, гордыня, закостенелый консерватизм, отсутствие желания 
учиться и учить, лжеучительство, плохие отношения с отцом, неуважение 
к традициям, жажда власти.
\endsubsection

\subsection{Рекомендации:}
\item Обучаться и обучать – не останавливаться.
\item Расширять своё сознание и не зациклиться на чём-то одном, 
научиться переключаться, видеть и принимать разные точки зрения.
\item Быть терпимее к людям.
\item Соблюдать законы и традиции, улучшая их при необходимости.
\endsubsection

\endsection

\section{id="6_genus"}{Родовые программы}

\subsection{Предназначение:}
Открывать и показывать миру любовь и красоту, нести радость и 
позитивное мышление, показывать людям как можно жить в лёгкости. 
Ваше тонкое чувствование прекрасного способно привнести в мир 
больше позитива и даже мира: «Красота спасёт мир».
\endsubsection

\subsection{Минусы:}
Нелюбовь к себе, зацикленность на внешних атрибутах, поверхностность, 
зависимость от других, нерешительность в выборе, обидчивость и 
ранимость.
На подсознательном уровне есть установка: что-то не так, меня не 
до любили, отношения у нас не такие, как я хочу. Они все время 
придираются к партнеру. Занимаются самокопанием. Как правило, 
сами разрушают отношения.
Из-за этого возникают аллергии и проблемы с кожей.
\endsubsection

\subsection{Рекомендации:}
\item Проработать самооценку и любовь к себе, свою обидчивость: 
практики, психолог, коуч.
\item Развивать своё чувство стиля и красоты.
\item Не судить людей «по одёжке», узнавать глубже, делать акцент 
на положительных качествах.
\item Не сомневаться в своём выборе, учиться его делать.
\item Не стремиться к идеальности.
\item Построить отношения – это Ваш двигатель во всех сферах.
Если Вы не построите отношения, то в сфере денег у Вас тоже будет 
ступор!
\endsubsection

\endsection

\section{id="7_genus"}{Родовые программы}

\subsection{Предназначение:}
Реализовать энергию воина, идущего вперёд лидера, умеющего вести 
за собой массы. То есть необходимо использовать свою энергию 
победителя, смелого, целеустремлённого, преодолевающего на своём 
пути все преграды.
\endsubsection

\subsection{Минусы:}
Лень и апатия, влекущая за собой застой в развитии. Из-за этого 
могут быть проблемы с костями, суставами, коленями. Могут случаться 
аварии.
Вспыльчивость, агрессия, враждебность, хаос в делаов, излишняя 
требовательность к себе и окружающим.
Вторая крайность – гонка. Страх что-то не успеть.
\endsubsection

\subsection{Рекомендации:}
\item Рекомендуется: спорт и ЗОЖ, они поддерживают уровень энергии 
и активности, жизненно необходимый этому аркану.
\item Следует занимать активную жизненную позицию.
\item Контролировать свои негативные эмоции.
\item Соблюдать определённый уровень безопасности: не увлекаться 
бездумным экстримом, не превышать скорость на дороге и т.п.
\item Реализовываться в разных сферах жизни.
\item Для Вас полезны будут поездки, переезды.
\endsubsection

\endsection

\section{id="8_genus"}{Родовые программы}

\subsection{Предназначение:}
Раскрыть свой талант видеть и интуитивно чувствовать причинно-
следственные законы Вселенной, удерживать внутренний баланс и 
объективность и транслировать это миру.
\endsubsection

\subsection{Минусы:}
Осуждение других, желание наказать и «нести меч» справедливости, 
деление мира на чёрное/белое, гордыня, гиперответсвенность и 
взваливание на себя лишнего или наоборот, безответственность, 
нечестность, меркантильность.
\endsubsection

\subsection{Рекомендации:}
\item Осознать, что не существует справедливости в бытовом смысле. 
Справедливость – понятие скорее вселенское, которое невозможно 
объять в пределах одной человеческой жизни.
\item Не осуждать, не судить других: необходимо понимание, что 
невозможно на 100% понять чужую точку зрения, не побывав в его шкуре.
\item Развивать в себе позитивное мышление, не циклиться на плохом.
\item Развивать интуицию, медитировать, сохранять эмоциональный 
баланс, научиться искренне прощать.
\item Не нарушать закон.
\endsubsection

\endsection

\section{id="9_genus"}{Родовые программы}

\subsection{Предназначение:}
Делиться своей мудростью с другими. Раскрыть свои целительские и 
экстрасенсорные способности, в т. ч. умение исцелять словом, 
развивать интуицию, чтобы передавать свои глубокие знания, 
познавать мир с разных сторон.
\endsubsection

\subsection{Минусы:}
Отшельничество, интеллектуальная гордыня, нежелание помогать 
людям, излишняя закрытость и холодность, синдром самозванца и 
комплекс неполноценности, аскетичность, склонность к депрессиям.
\endsubsection

\subsection{Рекомендации:}
\item Развивать свою интуицию.
\item Медитировать.
\item Изучать духовность, познавать разные знания, познавать мир 
с разных сторон.
\item Не уходить в себя, делиться и открываться перед людьми.
\item Помогать людям, не отказывать в помощи, делиться своей мудростью.
\endsubsection

\endsection

\section{id="10_genus"}{Родовые программы}

\subsection{Предназначение:}
Реализовать свою энергию потока, научиться плыть по течению, 
чувствуя его энергию. Научиться во всем видеть возможности и 
транслировать это в массы. Вам нужно научиться быть счастливчиком 
и это Ваша самая главная программа. Всё Вам в этой жизни дастся, 
Вы только доверьтесь и всему говорите «да».
\endsubsection

\subsection{Минусы:}
Гиперконтроль, неверие в Бога/высшие силы, нежелание прислушиваться 
к своей интуиции, пассивность.
\endsubsection

\subsection{Рекомендации:}
\item Развивать интуицию через различные практики. Научиться слышать 
и видеть знаки Вселенной, прислушиваться к себе.
\item Не зацикливаться на материальном. Деньги к Вам придут через 
отпускание контроля.
\item Иметь активную жизненную позицию, быть инициатором, используя 
свою чуйку. Но при этом в лёгкости, а не через неимоверные усилия.
\item Верить в себя, в свой успех.
\endsubsection

\endsection

\section{id="11_genus"}{Родовые программы}

\subsection{Предназначение:}
Используя Вашу внутреннюю и физическую силу, необходимо реализовать 
себя на благо других людей. Быть лидером, ведущим за собой, 
вдохновляющим других.
\endsubsection

\subsection{Минусы:}
Гордыня, агрессия, в т. ч. внутренняя, не выведенная экологичным 
образом наружу, жестокость, насилие над другими, подавление, 
работа на износ – трудоголизм.
\endsubsection

\subsection{Рекомендации:}
\item Научиться экологично сливать негативные эмоции, для этого 
идеально подойдёт тяжёлый и регулярный спорт: теннис, бокс, 
тренажёрный зал и пр.
\item Не сливать силы на негатив, развивать позитивное мышление.
\item Давать себе отдыхать, не доводить до изнеможения.
\item Помогать другим, заниматься благотворительностью.
\item Проработать гордыню, принятие.
\item Нельзя никому причинять вред.
\endsubsection

\endsection

\section{id="12_genus"}{Родовые программы}

\subsection{Предназначение:}
Реализовать свою энергию служения и нового знания. Доносить до 
людей новые идеи своим особым видением, при этом постоянно духовно 
развиваясь. Служить людям, проявляя искреннее милосердие, эмпатию 
и доброту. При этом важно не забывать про себя, найти баланс 
служение/жертвенность.
\endsubsection

\subsection{Минусы:}
Жертвенность, обесценивание себя, неумение отказывать, обидчивость, 
ощущение несправедливости, манипуляции.
\endsubsection

\subsection{Рекомендации:}
\item Научиться говорить «нет».
\item Учиться принимать, а не только отдавать.
\item Проработать самооценку.
\item Духовно развиваться, не закрываться от нового.
\item Помогать людям – найти свой путь служения, заниматься 
благотворительностью.
\item Делиться своим видением с миром.
\item Заниматься творчеством.
\endsubsection

\endsection

\section{id="13_genus"}{Родовые программы}

\subsection{Предназначение:}
Управлять переменами, показывать людям на своём примере важность 
изменений, новый путь трансформации, используя свои адаптационные 
способности. Вносить перемены в застоявшиеся дела.
\endsubsection

\subsection{Минусы:}
Агрессия, жестокость, страх перемен, пассивность, повисание, 
чрезмерный риск – пренебрежение жизнью.
\endsubsection

\subsection{Рекомендации:}
\item Внедрить в жизнь энергетические практики, гимнастику, для 
поддержания необходимого количества энергии.
\item Научиться отпускать старое, не зацепляться за ушедшее.
\item Научиться справляться со своей агрессией – медитации, практики, 
психолог.
\item Не бояться меняться и трансформироваться – найти в этом свою 
силу.
\endsubsection

\endsection

\section{id="15_genus"}{Родовые программы}

\subsection{Предназначение:}
Научиться и научить других усмирять свои пороки и освобождаться 
от соблазнов. Протягивать людям руку помощи, используя свою 
харизму и магнетичность и при этом, видя человеческие пороки, 
замечать в людях хорошее. Познать таким образом энергию любви и 
всепрощения.
\endsubsection

\subsection{Минусы:}
Эгоизм, цинизм, агрессия, желания наживы, манипуляции, зависимости, 
порочность, подавление других, алкоголь, несвободные партнеры, 
игры, лёгкие деньги, криминал. Это всё про бездуховность.
\endsubsection

\subsection{Рекомендации:}
\item Проработать зависимости.
\item Уважать других, задумываться об их чувствах.
\item Прорабатывать позитивное мышление – искать во всем и всех 
хорошее.
\item Помогать людям преодолеть сложности, используя свою 
проницательность и глубокое чувствование негатива.
\item Никогда не связывать себя с черной магией.
\endsubsection

\endsection

\section{id="16_genus"}{Родовые программы}

\subsection{Предназначение:}
Духовно пробудиться и расширить сознание, стать примером для 
других. Разрушать обременяющее и строить с нуля – мировоззрение, 
отношения и т.д. Нести высокие духовные цели и задачи.
\endsubsection

\subsection{Минусы:}
Агрессия, резкость, высокомерия, зависимости, склонность к 
разрушению, страх изменений, материализм.
В плюсе это человек, который создает что-то глубокое, 
трансформационное и духовное.
\endsubsection

\subsection{Рекомендации:}
\item Избегать разрушений: в отношениях, в работе.
\item Духовно развиваться, расширять своё сознание, быть готовым 
к духовной трансформации.
\item Прокачивать осознанность, личностное развитие, проработать 
гордыню, агрессию.
\item Научиться отпускать и не копить обиды на людей, жизнь.
\item Всегда искать новые идеи, пути и способы.
\item Учиться строить с нуля.
\item Укреплять своё тело, заниматься спортом.
\item Очищать своё пространство и тело.
\endsubsection

\endsection

\section{id="17_genus"}{Родовые программы}

\subsection{Предназначение:}
Раскрыть свои таланты, предъявить их миру и вдохновлять этим 
людей. Помогать другим в раскрытии талантов и продвижении. Быть 
духовным наставником.
\endsubsection

\subsection{Минусы:}
Неуверенность и неверие в себя, страхи, пессимизм, холодность, 
зависть, склонность к зависимостям, и предательству, гордыня, 
тщеславие.
\endsubsection

\subsection{Рекомендации:}
\item Прислушиваться к себе, к своим чувствам и желаниям.
\item Вырабатывать позитивное мышление.
\item Заниматься творчеством по душе.
\item Проработать неуверенность в себе.
\item Поддерживать свою энергетическую проводимость – медитации, 
энергетические практики, молитвы и др.
\item Не предавать и не красть чужие идеи, проекты.
\endsubsection

\endsection

\section{id="20_genus"}{Родовые программы}

\subsection{Предназначение:}
Реализовать свой дар целителя, провидца, проводника энергии и 
информации. Транслировать верную систему ценностей, через свой 
ораторский и духовный талант, мудрость и гибкость.
\endsubsection

\subsection{Минусы:}
Обидчивость, категоричность, осуждение, неумение прощать, 
непринятие рода и родителей, неуважение старших, отсутствие 
внутреннего стержня, желание всех подчинить своей воле.
\endsubsection

\subsection{Рекомендации:}
\item Наладить отношения и свои чувства к родным, всему Роду.
\item Научиться прощать.
\item Проработать обидчивость, категоричность, осуждение.
\item Развивать интуицию, использовать духовные практики, медитировать.
\item Изучать эзотерические, целительские направления.
\endsubsection

\endsection

\section{id="21_genus"}{Родовые программы}

\subsection{Предназначение:}
Нести высокую миротворческую миссию – концепцию согласия и 
духовного единения всех людей. Необходимо доносить до других 
идею о том, что все люди – братья и сестры, а Земля – наш общий 
дом.
\endsubsection

\subsection{Минусы:}
Воинственный настрой, нетерпимость, желание войны, нежелание 
развиваться духовно, замкнутость, апатия, желание жить за чужой 
счёт.
\endsubsection

\subsection{Рекомендации:}
\item Путешествовать, изучать иностранные языки.
\item Никого не судить, вырабатывать в себе терпимость.
\item Транслировать идеи миротворчества.
\item Познавать новое, учиться в том числе развивать духовную часть.
\item Мыслить масштабно, не бояться строить глобальные цели.
\item Не разделять людей ни по какому признаку.
\endsubsection

\endsection

\section{id="22_genus"}{Родовые программы}

\subsection{Предназначение:}
Привносить в этот мир новое, реализовываться через творчество. 
Открывать миру новые идеи и пути, транслировать внутреннюю 
лёгкость и свободу.
\endsubsection

\subsection{Минусы:}
Инфантильность, безответственность, капризность, глупость, 
зависимости, нетерпеливость, импульсивность, чрезмерная 
рискованность, слишком серьезное отношение к жизни.
\endsubsection

\subsection{Рекомендации:}
\item Путешествовать.
\item Брать определённый уровень ответственности и дисциплины.
\item Сохранить лёгкое отношение к жизни, адекватный авантюризм.
\item Заниматься творчеством.
\item Научиться легко отпускать, менять обстоятельства.
\item Избавляться от зависимостей.
\item Проработать страх идти в новое.
\endsubsection

\endsection
